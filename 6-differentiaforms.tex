\marginnote{There is a nice book by Guillemin and Haines which is all on differential forms \cite{book:guillemin-haine} and whose draft is freely available on the authors courses website.}
In the rest of the course we will focus on a particular class of tensors, which generalizes the differential one-forms that we studied on the cotangent bundle.
It should not be surprising then, that these will be called differential $k$-forms and that they will be alternating $(0,k)$-tensors, that is, skew-symmetric in all arguments.

Geometrically, they are not dissimilar from the forms you may have seen in multivariable calculus: a $k$-form takes $k$ vectors as arguments and computes the $k$-dimensional volume spanned by these $k$-vectors.
In this sense, they will be the key elements to define integration over $k$-dimensional manifolds, in the same way as one-forms and line integrals.

In addition to their role in integration, differential forms provide a framework for generalizing such diverse concepts from multivariable calculus as the cross product, curl, divergence, and Jacobian determinant. 

\section{Differential forms}

\begin{definition}
  Let $V$ be a real $n$-dimensional vector space.
  Let $S_k$ denote the \emph{symmetric group on $k$ elements}, that is, the group of permutations of the set $\{1,\ldots,k\}$.
  Recall that for any permutation $\sigma\in S_k$, the \emph{sign of $\sigma$}, denoted $\sgn(\sigma)$, is equal to $+1$ if $\sigma$ is even\footnote{It can be written as a composition of an even number of transpositions} and $-1$ is $\sigma$ is odd\footnote{It can be written as a composition of an odd number of transpositions}.

  \marginnote{In particular, exchanging two arguments changes the sign of $\omega$.}
  A tensor $\omega\in T_k^0(V)$, $0\leq k\leq n$, is called \emph{alternating} (or \emph{antisymmetric} or \emph{skew-symmetric}), if it changes sign whenever two of its arguments are interchanged, that is,
  for all $v_1, \ldots, v_k\in V$ and for any permutation $\sigma\in S_k$ it holds that
  \begin{equation}
    \omega(v_{\sigma(1)}, \ldots, v_{\sigma(k)}) = \sgn(\sigma) \omega(v_1, \ldots, v_k).
  \end{equation}
  The subspace of alternating tensors in $T_k^0(V)$ is denoted\footnote{Some authors also use $\Lambda^k(v^*)$ or $\Lambda_k(V^*)$ to denote the same space.} by $\Lambda^k \equiv \Lambda^k(V) \subset T_k^0(V)$ and its elements are called \emph{exterior forms}, \emph{alternating $k$-forms} or just  \emph{$k$-forms}.
  For $k=0$, we define $\Lambda^0 := T_0^0(V) := \R$.
\end{definition} 

\begin{exercise}\label{ex:propAlt}
  Show that the following are equivalent for a tensor $\omega\in T_k^0(V)$.
  \begin{enumerate}
    \item $\omega$ is alternating;
    \item $\omega$ is $0$ whenever two of its arguments are equal, that is, ${\omega(v_1, \ldots, w, \ldots, w, \ldots, v_k) = 0}$;
    \item $\omega(v_1, \ldots, v_k) = 0$ whenever the vectors $(v_1, \ldots, v_n)$ are linearly dependent.
  \end{enumerate}
\end{exercise}

\section{The wedge product}

\marginnote{You can find an interesting explanation of the wedge product, based on Penrose's book ``The road to reality'', \href{https://twitter.com/LucaAmb/status/1289244374996406273?s=20}{on a thread by @LucaAmb on Twitter}.}
If you remember, we said that the determinant was an example of a $T_n^0(R^n)$ tensor: an antisymmetric tensor nonetheless.
At the same time, the determinant of a $n\times n$ matrix, is the signed volume of the parallelotope spanned by the $n$ vectors composing the matrix.
We also saw that tensors can be multiplied with the tensor product, which gives rise to a graded algebra on the free sum of tensor spaces.
This leads naturally to the following definition.

\begin{definition}
  Let $V$ be a real $n$-dimensional vector space.
  Given $k$ covectors $\omega^1, \ldots, \omega^k\in T_1^0(V)$, their \emph{wedge product} (or \emph{exterior product}) $\omega^1\wedge\ldots\wedge\omega^k$ is defined by
  \begin{equation}
    \left(
      \omega^1\wedge\ldots\wedge\omega^k\,\mid\, v_1,\ldots,v_k
    \right) := \det\begin{pmatrix}
      \omega^1(v_1) & \cdots & \omega^1(v_k) \\
      \vdots & \ddots & \vdots \\
      \omega^k(v_1) & \cdots & \omega^k(v_k)
    \end{pmatrix} \qquad
    \forall v_1,\ldots,v_k\in V.
  \end{equation}
\end{definition}

Since the determinant changes sign when two of its columns are interchanged, $\omega^1\wedge\ldots\wedge\omega^k$ is alternating and thus an element of $\Lambda^k(V)$.
Similarly, since the determinant changes sign when two of its columns are interchanged, it holds that, for any $\sigma\in S_k$,
\begin{equation}\label{equiv:permut}
  \omega^{\sigma(1)}\wedge\ldots\wedge\omega^{\sigma(k)} = \sgn(\sigma) \omega^1\wedge\ldots\wedge\omega^k.
\end{equation}
That is, using Leibniz formula for the determinant\footnote{\cite[Equation (B.3)]{book:lee}}, we get
\begin{equation}\label{eq:detLeibniz}
  \omega^1\wedge\ldots\wedge\omega^k = \sum_{\sigma\in S_k}\sgn(\sigma) \omega^{\sigma(1)}\otimes\ldots\otimes\omega^{\sigma(k)}.
\end{equation}

According to Proposition~\ref{prop:tensorbasis} we have the basis representation
\begin{equation}
  \omega = \omega_{j_1, \ldots, j_k} e^{j_1}\otimes \cdots \otimes e^{j_k}
\end{equation}
in $T_k^0(V)$.
It would be convenient to have a similar basis representation on $\Lambda^k(V)$.

\begin{proposition}\label{prop:dimLkV}
  Let $V$ be a real $n$-dimensional vector space, let $(e^j)$ denote a basis for $V^*$.
  Then, for each $1\leq k \leq n$, the set of $k$-forms
  \begin{equation}
    E = \left\{
    e^{j_1}\wedge\cdots\wedge e^{j_k} \;\mid\; 1\leq j_1<\cdots<j_k\leq n
    \right\},
  \end{equation}
  forms a basis for the space $\Lambda^k(V) \subset T_k^0(V)$ of alternating $k$-forms.
  Therefore, if $1\leq k\leq n$
  \marginnote[3em]{In particular, $\Lambda^n = \Lambda^0 = \R$.}
  \begin{equation}
    \dim \Lambda^k(V) = \binom{n}{k} = \frac{n!}{k!(n-k)!},
  \end{equation}
  while if $k>n$, $\dim \Lambda^k(V) = 0$.
\end{proposition}
\begin{proof}
  The last point of Exercise~\ref{ex:propAlt} implies that there are no non-zero alternating $k$-tensors on $V$ if $k >\dim V$, since in that case every $k$-tuple of vectors would be dependent.
  For $k\leq n$ we need to show that $E$ spans $\Lambda^k(V)$ and its vectors are linearly independent.
  
  First of all, observe that by~\eqref{equiv:permut} all the wedge products ${e^{j_1}\wedge\ldots\wedge e^{j_k}\not\in E}$ are either zero (if two indices are repeated, i.e., a base vector appears twice) or are a linear multiple of an element of $E$ (the wedge product with the indices in the same set but in increasing order).

  Let now $\{e_i\}$ denote the basis for $V$ dual to $\{e^i\}$ and $\omega\in\Lambda^k$.
  By definition of alternating form, we have
  \begin{equation}\label{eq:altchar}
    \omega(e_{i_1}, \ldots, e_{i_k}) = \frac1{k!} \sum_{\sigma\in S_k}\sgn(\sigma) \omega\left(e_{i_{\sigma(1)}}, \ldots, e_{i_{\sigma(k).}}\right)
  \end{equation}
  Moreover, for any $v_1,\ldots,v_k\in V$ we have
  \marginnote[3em]{Don't forget, $1\leq i\leq n$.}
  \begin{equation}
    v_j = e^i(v_j) e_i, \quad j=1,\ldots, k.
  \end{equation}
  Therefore,
  \begin{align}
    \omega(v_1, \ldots, v_k)
    &=\omega\left(e^{i_1}(v_q) e_{i_1}, \ldots, e^{i_k}(v_k) e_{i_k}\right)\\
    &=e^{i_1}(v_1)\cdots e^{i_k}(v_k)\; \omega(e_{i_1}, \ldots, e_{i_k})\\
    &=e^{i_1}(v_1)\cdots e^{i_k}(v_k) \frac1{k!} \sum_{\sigma\in S_k}\sgn(\sigma)\; \omega\left(e_{i_{\sigma(1)}}, \ldots, e_{i_{\sigma(k).}}\right) \\
    \overset{\mbox{\small\eqref{eq:altchar}}}{}
    &= \frac1{k!} \left(e^{i_1}\otimes \cdots \otimes e^{i_k} \;\mid\; v_1, \ldots, v_k\right) \sum_{\sigma\in S_k}\sgn(\sigma)\; \omega\left(e_{i_{\sigma(1)}}, \ldots, e_{i_{\sigma(k)}}\right) \\
    \overset{[i_{\sigma(l)} \mapsto j_l]}{}
    &= \frac1{k!} \omega\left(e_{j_1}, \ldots, e_{j_k}\right) \sum_{\sigma\in S_k} \sgn(\sigma) \;\left(e^{j_{\sigma^{-1}(1)}}\otimes \cdots \otimes e^{j_{\sigma^{-1}(k)}} \;\mid\; v_1, \ldots, v_k\right)\\
    \overset{\mbox{\small\eqref{eq:detLeibniz}}}{}
    &= \frac1{k!} \omega\left(e_{j_1}, \ldots, e_{j_k}\right) \left( e^{j_1}\wedge \cdots \wedge e^{j_k} \;\mid\; v_1, \ldots, v_k\right) \\
    \overset{\mbox{\small dedup.}}{}
    &= \sum_{j_1=1}^{n-k+1}\sum_{j_2=j_1+1}^{n-k+2}\cdots \sum_{j_k=j_{k-1}+1}^{n} \omega\left(e_{j_1}, \ldots, e_{j_k}\right) \left(e^{j_1}\wedge \cdots \wedge e^{j_k} \;\mid\; v_1, \ldots, v_k\right).
  \end{align}
  That is,
  \begin{equation}
    \omega = \sum_{j_1=1}^{n-k+1}\sum_{j_2=j_1+1}^{n-k+2}\cdots \sum_{j_k=j_{k-1}+1}^{n} \omega_{j_1, \ldots, j_k} e^{j_1}\wedge \cdots \wedge e^{j_k},
  \end{equation}
  where $\omega_{j_1, \ldots, j_k} = \omega\left(e_{j_1}, \ldots, e_{j_k}\right)$, in analogy with Proposition~\ref{prop:tensorbasis}.
\end{proof}

\begin{remark}
  There are multiple alternative definitions of the wedge product, which are equivalent up to a multiplicative factor.
  Be careful when you consult the literature to check the conventions used.

  The convention that we are using here is usually called the determinant convention and is usually the most convenient for computations. The name stems from the fact that if $(e^i)$ denotes the standard basis for ${(\R^n)}^*$, then for some vectors $v_1, \ldots, v_n \in\R^n$,
  \begin{equation}
    \det(\LaTeXunderbrace{v_1 \,\cdots\, v_n}_{\parbox{1.5cm}{\tiny $n\times n$ matrix with\\$v_i$ as columns}}) = e^1\wedge\cdots\wedge e^n(v_1,\ldots,v_n).
  \end{equation}
\end{remark}

As you could see from the previous proof, Einstein notation can help but only to a certain extent.
There is an extra bit of notation, also common in higher-dimensional analysis, that can be often convenient when working with many indices.
\begin{notation}
  Given a positive integer $k$, an ordered\footnote{That is, $1\leq i_1<\cdots<i_k\leq n$.} $k$-tuple $I=(i_1, \ldots, i_k)$ of positive integers is called \emph{multi-index of length $k$}.
  If $I$ is such a multi-index and $\sigma\in S_k$ is a permutation of $\{1,\ldots,k\}$, then we denote $I_\sigma := (i_{\sigma(1)}, \ldots, i_{\sigma(k)})$.
  Defining $e^I := e^{i_1}\wedge\cdots\wedge e^{i_k}$, we finally get the more compact notation $\omega = \omega_I e^I$.
\end{notation}

In general, the tensor product $\omega\otimes\nu\in T_{k+h}^0(V)$ of alternating forms $\omega\in\Lambda^k$ and $\nu\in\Lambda^{h}$ is not an alternating form.
The following proposition gives us a tool to define an exterior product of alternating forms.

\begin{proposition}
  Let $\Alt_k: T_k^0(V)\to \Lambda^k(V)$ be the map defined by
  \begin{equation}
    (\Alt_k\tau)(v_1,\ldots,v_k) := \frac1{k!} \sum_{\sigma\in S_k} \sgn(\sigma) \tau(v_{\sigma(1)}, \ldots, v_{\sigma(k)}),
    \qquad \forall v_1,\ldots,v_k \in V.
  \end{equation} 
  Then $\Alt_k$ is a linear projection and the following holds:
  \begin{equation}
    \omega^1 \wedge \cdots \wedge \omega^k = k! \Alt_k(\omega^1 \otimes \cdots \otimes \omega^k).
  \end{equation}
\end{proposition}
\begin{proof}
  Linearity is there by construction, we need to check that $\Alt_k$ is a projection.
  This follows from a direct computation of its idempotence:
  \begin{align}
    (\Alt_k \Alt_k \tau)(v_1,\ldots, v_k)
    &= \frac1{k!k!}\sum_{\sigma,\sigma'\in S_k}\sgn(\sigma)\sgn(\sigma') \tau\left(v_{\sigma'\circ\sigma(1)}, \ldots, v_{\sigma'\circ\sigma(1)}\right)\\
    \overset{\widetilde \sigma = \sigma'\circ\sigma}{}
    &=\frac1{k!k!}\sum_{\sigma,\eta\in S_k}\sgn(\eta) \tau\left(v_{\eta(1)}, \ldots, v_{\eta(1)}\right) \\
    &=\frac1{k!}\sum_{\eta\in S_k}\sgn(\eta) \tau\left(v_{\eta(1)}, \ldots, v_{\eta(1)}\right) \\
    &=(\Alt_k \tau)(v_1,\ldots, v_k),
  \end{align}
  where we used the fact that $\eta$ runs over all $S_k$, as $\sigma$ does.
  Then the result follows from~\eqref{eq:detLeibniz}.
\end{proof}

As we were saying, now we can take the tensor product of two forms $\omega\otimes\nu$ and use the antisymmetrisation $\Alt_{k+h}$ to to project it onto the antisymmetric subspace $\Lambda^{k+h}$ of $T_{k+h}^0(V)$.

\begin{definition}[Wedge product of alternating forms]
  We can extend the wedge product (or exterior product) to alternating forms by defining, for any $k,h\in\N$,
  \begin{align}
    \wedge : \Lambda^k\times\Lambda^h &\to \Lambda^{k+h}\\
    (\omega, \nu) &\mapsto \omega\wedge\nu := \frac{(k+h)!}{k!h!} \Alt_{k+h}(\omega\otimes\nu).
  \end{align}
\end{definition}

\begin{example}
  The wedge product of two $1$-forms $\omega$ and $\nu$ is
  \begin{equation}
    \omega\wedge\nu = 2\Alt_2(\omega\otimes\nu) = 2 \frac12 (\omega \otimes \nu - \nu \otimes \omega).
  \end{equation}
\end{example}

\begin{exercise}
  Compute the wedge product of three $1$-forms.
\end{exercise}

\begin{proposition}
  The wedge product has the following properties.
  \begin{enumerate}
    \item (associative) $(\omega^1\wedge\omega^2)\wedge\omega^3 = \omega^1\wedge(\omega^2\wedge\omega^3)$ for $\omega^i\in\Lambda^{k_i}$, $i=1,\ldots, 3$;
    \item (distributive) $(\omega^1+\omega^2)\wedge\omega^3 = \omega^1\wedge\omega^3+\omega^2\wedge\omega^3$ for $\omega^1,\omega^2\in\Lambda^{k}$ and $\omega^3\in\Lambda^h$;
    \item (distributive) $\omega^1\wedge(\omega^2+\omega^3) = \omega^1\wedge\omega^2+\omega^1\wedge\omega^3$ for $\omega^1\in\Lambda^{k}$ and $\omega^2,\omega^3\in\Lambda^h$;
    \item $\omega^1\wedge\omega^2 = {(-1)}^{hk}\omega^2\wedge\omega^1$ for $\omega^1\in\Lambda^{k}$ and $\omega^2\in\Lambda^h$.
  \end{enumerate}
\end{proposition}
\begin{exercise}
  Prove the proposition.\\
  \textit{\small Hint: keep in mind the tricks used in the proof of the previous propositions.}
\end{exercise}

\begin{exercise}\label{ex:zeroform}
  Let $V$ be a real $n$-dimensional vector space.
  Prove that if an $n$-form $\omega$ vanishes on a basis $e_1,\ldots,e_n$ for $V$, then $\omega$ is the zero $n$-form on $V$.
\end{exercise}

\begin{remark}
  As for tensors, if we define the $2^n$-dimensional vector space
  \begin{equation}
    \Lambda(V) = \bigoplus_{k=0}^n \Lambda^k(V),
  \end{equation}
  then the wedge product turns it into an associative, anticommutative\footnote{A graded algebra is anticommutative is the product satisfies a relation of the form $uv = {(-1)}^{kh}vu$, where $u$ and $v$ are in the spaces of the gradation with indicex $k$ and $h$ respectively.} graded algebra, called \emph{exterior algebra of $V$}.
\end{remark}

\section{The interior product}

There is an extremely important operation that relates vectors with alternating tensors.

\begin{definition}
  Let $V$ be a real $n$-dimensional vector space.
  For each $v\in V$, the \emph{interior multiplication by $v$} is a contraction of a $k$-form by $v$, that is, the linear map $\iota_v:\Lambda^{k}(V)\to \Lambda^{k-1}(V)$ defined\footnote{Another common notation for the same operation is $v \iprod \omega$.} by
  \begin{equation}
    \iota_v\omega(w_1,\ldots,w_{k-1}) = \omega(v,w_1,\ldots,w_{k-1})
    \quad \forall w_1,\ldots,w_{k-1} \in V.
  \end{equation}
\end{definition}

In other words, $\iota_v\omega$ is obtained from $\omega$ by inserting $v$ into ``the first slot''.
By convention $\iota_v\omega = 0$ if $\omega\in\Lambda^0(V)$.

\begin{lemma}\label{lemma:intprod}
  Let $V$ be a real $n$-dimensional vector space and $v\in V$.
  Then the following hold.
  \begin{enumerate}
    \item $\iota_v\circ \iota_v = 0$ and $\iota_v\circ\iota_u = -\iota_u\circ\iota_v$;
    \item if $\omega\in\Lambda^k(V)$ and $\eta\in\Lambda^h(V)$,
    \begin{equation}
      \iota_v(\omega\wedge\eta) = (\iota_v\omega)\wedge\eta + {(-1)}^k\omega\wedge(\iota_v\eta).
    \end{equation}
  \end{enumerate}
\end{lemma}
\begin{exercise}[\textit{[homework 4]}]
  Prove the Lemma.
\end{exercise}

\section{Differential forms on manifolds}

It is time to turn our attention back to smooth manifolds.
Let $M$ be a $n$-dimensional smooth manifold, recall that we had defined the tensor fields $\cT_s^r(M)$ as the sections of $(r,s)$-tensor bundles $T_s^r(M)$ over $M$.
The subset of $T_k^0(M)$ consisting of alternating $k$-tensors is denoted by $\Lambda^k(M):= \bigsqcup_{p\in M} \{p\}\times \Lambda^k(T_p M)$.
\begin{definition}
  The sections of $\Lambda^k(M)$ are called \emph{differential $k$-forms}, or just $k$-forms: these are smooth tensor fields whose values at each point are alternating tensors. The integer $k$ is called the \emph{degree} of the $k$-form.
  
  We denote the vector space of smooth $k$-forms by
  \begin{equation}
    \Omega^k(M) = \Gamma(\Lambda^k(M)).
  \end{equation}
  The wedge product $\wedge:\Omega^k(M)\times\Omega^h(M)\to \Omega^{k+h}(M)$ of differential forms is defined pointwise as ${(\omega\wedge\nu)}_p = \omega_p\wedge\nu_p$.
\end{definition}

\begin{example}
  \begin{enumerate}
    \item A $0$-form is just a function $f\in C^\infty(M)$ and $1$-forms are just the covector fields $\omega\in\cT_1^0(M) = \fX^*(M)$ on $M$.
    \item Let $M=\R^3$, then both $\cos(xy)dy\wedge dz$ and $dx\wedge dy - y dx\wedge dz + e^x/(x^2+y^2+1) dz\wedge dy$ are examples of smooth $2$-forms.
    \item Every $3$-form in $\R^3$ is a continuous real-valued function times $dx\wedge dy\wedge dz$.
  \end{enumerate}
\end{example}

\begin{remark}
  If we define
  \begin{equation}
    \Omega^*(M) = \bigoplus_{k=0}^n \Omega^k(M),
  \end{equation}
  then the wedge product turns $\Omega^*(M)$ into an associative, anticommutative graded algebra.    
\end{remark}

The following theorem gives a computational rule for pullbacks of differential forms similar to the ones we developed for covector fields and arbitrary tensor fields earlier.
In fact, it is a direct consequence of our previous observations.

\begin{theorem}\label{thm:pullbacksdifferentialforms}
  Let $F: M\to N$ be a smooth map between smooth manifolds.
  Let $\omega\in\Omega^k(N)$ and $\nu\in\Omega^h(N)$.
  Then,
  \begin{equation}
    F^*(\omega\wedge\nu) = F^*\omega \wedge F^*\nu,
  \end{equation}
  and, if $(x^i)$ denote some local coordinates on $U\subset N$, locally
  \begin{equation}
    F^*\left(\omega_J dx^J\right) = (\omega_{j_1,\ldots, j_k}\circ F) d(x^{j_1}\circ F)\wedge\cdots\wedge d(x^{j_k}\circ F).
  \end{equation}
\end{theorem}
\begin{exercise}%[\textit{[homework 4]}]
  Prove the theorem.
\end{exercise}

\begin{example}
  Let $F:\R^2\to\R^3$ be defined by $F(u,v) = (u^2,v,u-v^2)$ and let $\omega = y dz\wedge dx + z dx\wedge dy$ on $\R^3$.
  We can apply the previous theorem to compute $F^*\omega$:
  \begin{align}
    F^*\omega &= v d(u-v^2)\wedge d(u^2) + (u-v^2) d(u^2)\wedge dv \\
    &= v (du-2vdv)\wedge (2 u du) + (u-v^2) (2u du)\wedge dv\\
    &= -4uv^2 dv\wedge du + 2u(u-v^2) du\wedge dv \\
    &= 2u (u + 3v^2) du \wedge dv,
  \end{align}
  where we used that $du\wedge du =0$ and $du\wedge dv = -dv\wedge du$.
\end{example}

Of course, the same technique can also be used to compute the expression for a differential form in another smooth chart.

\begin{example}
  Let $\omega = dx\wedge dy$ on $\R^2$.
  Consider the polar coordinates $(x,y)\mapsto (\rho\cos(\theta),\rho\sin(\theta))$, then
  \begin{align}
    dx\wedge dy &= d(\rho\cos\theta)\wedge d(\rho\sin\theta) \\
    &= (\cos\theta d\rho -\rho\sin\theta d\theta)\wedge (\sin\theta d\rho + \rho\cos\theta d\theta) \\
    &= \rho d\rho\wedge d\theta.
  \end{align}

  I am very confident that it is not the first time that you see the equation above\ldots
\end{example}

\begin{exercise}
  Let $(x^i)$ and $(y^i)$ are two different local coordinates on some open $V\subset M$.
  Show that the following identity holds:
  \begin{equation}
    dy^1\wedge dy^n = \det\left(\frac{\partial y^j}{\partial x^i}\right) dx^1\wedge dx^n.
  \end{equation}
\end{exercise}

The previous exercise is a particular case of the following statement.

\begin{proposition}\label{prop:wedgeToJDet}
  Let $F:M\to N$ be a smooth map between $n$-manifolds.
  Let $(x^i)$ and $(y^i)$ denote, respectively, smooth coordinates on open subsets $U\subseteq M$ and $V\subseteq N$.
  Let $u$ be a continuous real-valued function on $V$.
  Then, on $U\cap F^{-1}(V)$, the following holds:
  \begin{equation}
    F^*(u\, dy^1\wedge\cdots\wedge dy^n)
    = (u\circ F) (\det DF) dx^1\wedge\cdots\wedge dx^n,
  \end{equation}
  where $DF$ represents the Jacobian matrix of $F$ in these coordinates.
\end{proposition}

\begin{exercise}\textit{[homework 4]}
  Prove the Proposition~\ref{prop:wedgeToJDet}.\\
  \textit{\small Hint: look at Theorem~\ref{thm:pullbacksdifferentialforms}.}
\end{exercise}

\begin{exercise}
  Let $f: \R^2 \to \R^3$ be defined by
  \begin{equation}
    f(x, y) = (x^2, y^2, xy).
  \end{equation}
  Compute the pullback $f^*\omega$ where $\omega$ is the form:
  \begin{enumerate}
    \item $\omega = y dy + z dz$;
    \item $\omega = x dy \wedge dx$;
    \item $\omega = dx \wedge dy\wedge dz$.
  \end{enumerate}
\end{exercise}

Of course, also the interior product extends naturally to vector fields and differential forms, simply by letting it act pointwise: if $X\in\fX(M)$ and $\omega\in\Omega^k(M)$, then the $k-1$-form $\iota_X\omega\equiv X\iprod\omega$ is defined by $(X\iprod\omega)_p = X_p \iprod \omega_p$.

\begin{exercise}
  Let $X\in\fX(M)$. Prove the following statements.
  \begin{enumerate}
    \item If $\omega$ is a smooth differential form, then $\iota_X\omega$ is smooth.
    \item The map $\iota_X:\Omega^k(M)\to\Omega^{k-1}(M)$ is linear over $C^\infty(M)$.
  \end{enumerate}
\end{exercise}

\section{Exterior derivative}

We already saw in the previous chapters that the differential of a function $f\in\Omega^0(M)$ can be thought as a $1$-form $df\in\Omega^1(M)$.
We are finally ready to generalise the concept to a map $d:\Omega^k(M)\to\Omega^{k+1}(M)$.
You have already seen most of this in the context of multivariable analysis, however it is good to repeat it to set the notational conventions.

\begin{definition}
  \marginnote{The same exact definition holds with $\R^n$ replaced by $\cH^n$.}
  Let $\omega\in\Omega^k(U)$ for some open subset $U\subset\R^n$ and let $(e^i)$ denote the standard basis for ${(\R^n)}^*$. If
  \begin{equation}
    \omega = \omega_I de^I, \quad \omega_I\in C^\infty(U),
  \end{equation}
  then its \emph{exterior deriative} $d\omega \in\Omega^{k+1}(U)$ is defined by
  \begin{equation}
    d\omega := d\omega_I \wedge de^I = \sum_{1\leq i_1 < \cdots < i_k \leq n} d\omega_{i_1,\ldots,i_k}\wedge de^{i_1}\wedge\cdots\wedge de^{i_k},
  \end{equation}
  where $d\omega_I$ is the differential of the function $\omega_I$.
\end{definition}

\begin{example}
For a smooth $0$-form\footnote{A real valued function} $f$, we have that $df = \frac{\partial f}{\partial x^i}dx^i$.
If $\omega$ is a $1$-form, this instead becomes
\begin{align}
  d(\omega_j dx^j)
  &= \frac{\partial \omega_j}{\partial x^i} dx^i \wedge dx^j \\
  &= \sum_{i<j} \frac{\partial \omega_j}{\partial x^i} dx^i \wedge dx^j  + \sum_{i>j} \frac{\partial \omega_j}{\partial x^i} dx^i \wedge dx^j \\
  &= \sum_{i<j} \left(\frac{\partial \omega_j}{\partial x^i} - \frac{\partial \omega_i}{\partial x^j} \right) dx^i\wedge dx^j,
\end{align}
consistently with our previous definitions.
\end{example}

\begin{definition}
  Let now $M$ be a smooth $n$-manifold and $\omega\in\Omega^k(M)$.
  Let $(U,\varphi)$ denote a chart on $U\subset M$ with local coordinates $(x^i)$.
  Then, the \emph{exterior derivative} is defined locally as $d\omega|_U := \varphi^*d(\varphi_* \omega)$, that is, for
  \begin{equation}
    \omega|_U = \omega_I dx^I, \quad \omega_I \in C^\infty(M),
  \end{equation}
  we define
  \begin{equation}\label{eq:localdw}
    d\omega|_U := d\omega_I \wedge dx^I.
  \end{equation}
\end{definition}

This local definition immediately extends to global one via the following theorem.

\begin{theorem}\label{thm:differentialpushforward}
  Let $M$ be a smooth $n$-manifold, $(U,\varphi)$ a chart on $U\subset M$ and $F:M\to N$ a diffeomorphism between smooth manifolds.
  Then, for $\omega\in\Omega^k(N)$, we have $F^*(d\omega|_{F(U)}) = dF^*\omega|_U$.
\end{theorem}
\begin{proof}
  Let $(x^i)$ denote the local coordinates of $\varphi$ and let 
  $(\widetilde U, \widetilde\varphi) := (F(U), \varphi\circ F^{-1})$ be the corresponding chart on $F(U)\subset N$ with local coordinates $(y^i)$ on $N$.
  Locally, $\omega = \omega_I dy^I$, thus we get
  \begin{align}
    d F^* \omega|_U
    &= d\left( \sum_{I=(i_1,\ldots,i_k)}(\omega_I\circ F) F^*(dy^{i_1})\wedge\cdots\wedge F^*(dy^{i_k})\right) \\
    &= d\left( \sum_{I=(i_1,\ldots,i_k)}(\omega_I\circ F) d(y^{i_1} \circ F)\wedge\cdots\wedge d(y^{i_k}\circ F)\right) \\
    &= d\left( \sum_{I=(i_1,\ldots,i_k)}(\omega_I\circ F) dx^{i_1}\wedge\cdots\wedge dx^{i_k}\right) \\
    &= \sum_{I=(i_1,\ldots,i_k)} d(\omega_I\circ F)\wedge dx^{i_1}\wedge\cdots\wedge dx^{i_k} \\
    &= \sum_{I=(i_1,\ldots,i_k)} F^* d(\omega_I)\wedge d(y^{i_1}\circ F)\wedge\cdots\wedge d(y^{i_k}\circ F) \\
    &= \sum_{I=(i_1,\ldots,i_k)} F^* d(\omega_I)\wedge F^*(dy^{i_1})\wedge\cdots\wedge F^*(dy^{i_k}) \\
    &= F^*(d\omega|_{F(U)}),
  \end{align}
  where we repeatedly applied Proposition~\ref{thm:pullbacksdifferentialforms} and Exercise~\ref{ex:propdiff} to swap pushforwards and differentials.
\end{proof}

\begin{corollary}
  Let $M$ be a smooth $n$-manifold and $(U_i, \varphi_i)$, $i=1,2$, two charts on $M$.
  Then, for $\omega\in\Omega^k(M)$, the following holds
  \begin{equation}
    \varphi_1^*\left(d(\varphi_{1*}\omega)_{\varphi_1(U_1\cap U_2)}\right) =
    \varphi_2^*\left(d(\varphi_{2*}\omega)_{\varphi_2(U_1\cap U_2)}\right).
  \end{equation}
  Therefore, the exterior derivative $d\omega\in\Omega^k(M)$ is uniquely defined by the local definition~\eqref{eq:localdw}.
\end{corollary}
\begin{proof}
  Follows from Theorem~\ref{thm:differentialpushforward} applied with $F = \varphi_1\circ\varphi_2^{-1} : \varphi_2(U) \to \varphi_1(U)$, where $U=U_1\cap U_2$.
  Indeed, since by the chain rule $F^* = (\varphi_2^{-1})^*\varphi_1^* = \varphi_{2*}(\varphi_1^{-1})_*$, we have
  \begin{align}
    \varphi_1^*\left(d(\varphi_{1*}\omega)_{\varphi_1(U)}\right)
    &= \varphi_2^* (\varphi_2^{-1})^* \varphi_1^*\left(d(\varphi_{1*}\omega)_{\varphi_1(U)}\right) \\
    &= \varphi_2^* F^*\left(d(\varphi_{1*}\omega)_{\varphi_1(U)}\right) \\
    &= \varphi_2^* \left(d(F^*\varphi_{1*}\omega)_{\varphi_2(U)}\right) \\
    &= \varphi_2^* \left(d(\varphi_{2*}\omega)_{\varphi_2(U)}\right).
  \end{align}
\end{proof}

\begin{exercise}\label{ex:smoothpushforward}
  Let $F:M\to N$ be a smooth map between smooth manifolds and $\omega\in\Omega^k(N)$, then
  \begin{equation}
    F^* (d\omega) = d(F^* \omega).
  \end{equation}
\end{exercise}

\begin{lemma}
  The exterior derivative satisfies the following properties.
  For all $\omega,\omega_1,\omega_2\in\Omega^k(M)$, $\nu\in\Omega^h(M)$ and $f\in C^\infty(M)$,
  \begin{enumerate}[(i)]
    \item $d(\omega_1 + \omega_2) = d\omega_1 + d\omega_2$;
    \item $d(f\omega) = df\wedge \omega + f d\omega$;
    \item $d(\omega\wedge\nu) = d\omega\wedge \nu + (-1)^k \omega\wedge d\nu$;
    \item $d(d\omega) = 0$.
  \end{enumerate}
\end{lemma}
\begin{proof}
  The first two properties immediately follow from the definition.
  Property $(iii)$ follows observing that to compare the two sides of the equation, one needs to keep commuting the exterior derivatives of coefficients of $\nu$ through the $k$-form $\omega$.

  The final property follows from the commutativity of the partial derivatives. Indeed, locally on a chart on $U\subset M$ with coordinates $(x^i)$, one has
  \begin{align}
    d(d\omega|_U) &= \frac{\partial^2 \omega_I}{\partial x^k\partial x^j} dx^k\wedge dx^j \wedge dx^I \\
    &= \sum_{j<k} \left(\frac{\partial^2 \omega_I}{\partial x^k\partial x^j}  - \frac{\partial^2 \omega_I}{\partial x^j\partial x^k}\right)dx^k\wedge dx^j \wedge dx^I \\
    &= 0.
  \end{align}
\end{proof}

\begin{exercise}
  Compute the exterior derivatives of the following differential forms on $\R^3$:
  \begin{enumerate}
    \item $x dy \wedge dz$;
    \item $x dy - y dx$;
    \item $e^{-f}df$ where $f = x^2 + y^2 + z^2$;
    \item $x dx + y dy + z dz$;
    \item $x dy\wedge dz - y dx\wedge dz + z dx\wedge dy$.
  \end{enumerate}
\end{exercise}

\begin{exercise}
  Solve the equation $d \nu = \omega$ for $\nu \in \Omega^1(\R^3)$ where $\omega$ is the $2$-form:
  \begin{enumerate}
    \item $d y \wedge d z$;
    \item $y\; d y \wedge d z$;
    \item $x^2 + y^2 \; d x \wedge d y$;
    \item $\cos(x)\; d x \wedge d z$.
  \end{enumerate}
\end{exercise}

Let $N\subset M$ a submanifold and $i:N\hookrightarrow M$ the corresponding injection.
For $\omega\in\Omega^k(M)$, we call $i^*\omega \in \Omega^k(N)$ the restriction of $\omega$ to $N$.
Exercise~\ref{ex:smoothpushforward}, then, implies that restriction and exterior derivative commute, that is, $i^*d\omega = d(i^*\omega)$.

\begin{example}[Exterior derivatives and vector calculus in $\R^3$]
  Let $M=\R^3$. Any smooth $1$-form $\omega\in\Omega^1(\R^3)$ can be written as
  \begin{equation}
    \omega = P dx + Q dy + R dz
  \end{equation}
  for some smooth functions $P,Q,R\in C^\infty(\R^3)$.
  Using the properties of wedge product, we can compute its exterior derivative and get the two form
  \begin{align}
    d\omega &= \left(\frac{\partial P}{\partial x} dx + \frac{\partial P}{\partial y} dy + \frac{\partial P}{\partial z} dz \right) \wedge dx \\
      &\quad+\left(\frac{\partial Q}{\partial x} dx + \frac{\partial Q}{\partial y} dy + \frac{\partial Q}{\partial z} dz \right) \wedge dy \\
      &\quad+\left(\frac{\partial R}{\partial x} dx + \frac{\partial R}{\partial y} dy + \frac{\partial R}{\partial z} dz \right) \wedge dz \\
    &= \left(\frac{\partial Q}{\partial x} - \frac{\partial P}{\partial y}\right) dx \wedge dy + 
    \left(\frac{\partial R}{\partial x} - \frac{\partial P}{\partial z}\right) dx \wedge dz + \left(\frac{\partial R}{\partial y} - \frac{\partial Q}{\partial z}\right) dy \wedge dz.
  \end{align}
  Similarly, an arbitrary $2$-form $\eta\in\Omega^2(\R^3)$ can be written as
  \begin{equation}
    \eta = u dx\wedge dy + v dx\wedge dz + w dy\wedge dz,
  \end{equation}
  and one can check (do it!)
  \begin{equation}
    d\eta = \left(\frac{\partial u}{\partial z}-\frac{\partial v}{\partial y} + \frac{\partial w}{\partial x}\right) dx\wedge dy \wedge dz.
  \end{equation}
  If you compare the results we obtained above with the gradient ($\nabla$), divergence ($\nabla\cdot$) and curl ($\nabla\times$) from multivariable analysis, you would likely notice that the components of the $2$-form $\d\omega$ are exactly the componentes of the curl of the vector field with components $(P, Q, R)$.
  Similarly, the formula for the divergence will look very close to the formula for $d\eta$. 
  What is going on?
  
  The standard euclidean metric on $\R^n$ is the metric associated to the metric tensor\footnote{Cf. Definition~\ref{def:metric}.} $g_{ij} = \delta_{ij}$.
  We can use the musical isomorphisms\footnote{Cf. Example~\ref{ex:musicaliso}.} to identify vector fields and $1$-forms, obtaining for the components with respect to cartesian coordinates that $v_i = v^i$.
  
  Moreover, the interior multiplication yields another map $\beta: \fX(\R^3)\to\Omega^2(\R^3)$ defined by $\beta(X) = \iota_X (dx\wedge dy\wedge dz)$, which is linear over $C^\infty(\R^3)$ (why?) and, thus, corresponds to a smooth bundle homomorphism from $T\R^3$ to $\Lambda^2(\R^3)$ (why?).

  In a similar fashion, we can also define a smooth bundle isomorphism $\bigstar: C^\infty(\R^3) \to \Omega^3(\R^3)$ via
  \begin{equation}
    \bigstar(f) = f dx\wedge dy\wedge dz.
  \end{equation}

  %If $f\in C^\infty(\R^3)$ and $v\in\cT_0^1(\R^3)$,
  We can use the exterior derivatives to observe that the following diagram commutes
  \begin{equation}\label{diag:comm:r3ops}
    \begin{tikzcd}
      C^\infty(\R^3) \arrow[d, "\id"] \arrow[r, "\nabla"] &
      \fX(\R^3) \arrow[d, "{}^\flat"] \arrow[r, "\nabla\times"] &
      \fX(\R^3) \arrow[d, "\beta"] \arrow[r, "\nabla\cdot"] &
      C^\infty(\R^3) \arrow[d, "\bigstar"] \\
      \Omega^0(\R^3) \arrow[r, "d"] &
      \Omega^1(\R^3) \arrow[r, "d"] &
      \Omega^2(\R^3) \arrow[r, "d"] &
      \Omega^3(\R^3)
    \end{tikzcd}.
  \end{equation}

  The interest and need to generalize the operations of vector calculus in $\R^3$ to higher dimensional spaces have been one of the drives to develop the theory of differential forms.
  In particular, the curl is well-defined as an operator on vector fields only in dimension $3$, while with the exterior derivative we can now generalize its meaning in all dimensions.
\end{example}

\begin{exercise}
  Show that the diagram~\eqref{diag:comm:r3ops} commutes; for example,
  \begin{equation}
    df = \frac{\partial f}{\partial x^i} = (\nabla f)_i dx^i = (\nabla f)^\flat.
  \end{equation}
  Use the diagram to give a quick proof that $(\nabla\times)\circ \nabla \equiv 0$ and that $(\nabla\cdot) \circ (\nabla\times) \equiv 0$ (physically this last identity implies that magnetic fields are divergence free).
\end{exercise}

\begin{exercise}\label{exe:symplectic}
  Let $V$ a vector space of dimension $k$.
  A symplectic form on $V$ is an element $\omega\in\Lambda^2(V)$ which is non-degenerate in the sense that $\iota_v(\omega) = 0$ if and only if $v=0$.
  Cf. Definition~\ref{def:metric}.
  A \emph{symplectic manifold} is a smooth manifold $M$ equipped with a closed differential 2-form $\omega$ such tht $\omega_q$ is a symplectic form on $T_q M$ for every $q\in M$.
  \begin{enumerate}
    \item Prove that if a symplectic form exists, then $k=2n$ for some $n\in\N$, i.e., it must be an even number.
    \item Let $M$ be a smooth manifold. Define a $1$-form $\eta\in\Omega^1(T^*M)$ on the cotangent bundle of $M$ as
    \begin{equation}
      %\lambda_{(q,p)}(\xi) = p(d\pi_{(q,p)} \xi),
      \lambda_{(q,p)} = d\pi_{(q,p)}^*p,
      \quad
      q\in M, \;
      p\in T_q^*M,
      %\;\xi\in T_{(q,p)}(T^*M), 
    \end{equation}
    where $\pi:T^*M\to M$ is the projection to the base.
    Show that $\omega := d\lambda$ is a symplectic form on $T^* M$, that is, every cotangent bundle is a symplectic manifold.
  \end{enumerate}

  For example, $\omega = \sum_{i=1}^n \alpha^i\wedge \alpha^{i+n}\in\Omega^2(\R^{2n})$ is a symplectic form and plays a central role in classical mechanics. There, one usually calls $(\alpha^{n+1},\ldots,\alpha^{2n})$ the \emph{position coordinates} and $(\alpha^{1},\ldots,\alpha^{n})$ the \emph{momentum coordinates}.
  \begin{enumerate}
    \item[3.] Show that for $n=2$, $\omega \wedge \omega = -2 \alpha^1\wedge\alpha^2\wedge\alpha^3\wedge\alpha^4$.

    \item[4.] Generalize the previous computation to show that
    \begin{equation}
      \bigwedge_{k=1}^{2n} \alpha^k := \alpha^1\wedge\cdots\wedge\alpha^{2n} = \frac{(-1)^{\binom{n}{2}}}{n!} \LaTeXunderbrace{\omega\wedge\cdots\wedge\omega}_{n\mbox{ times}} =: \frac{(-1)^{\binom{n}{2}}}{n!} \wedge^{n} \omega.
    \end{equation}
  \end{enumerate}
\end{exercise}

\section{Lie derivative}

\begin{definition}
The \emph{Lie derivative} of a differentiable function $f:M\to\R$ on a smooth manifold $M$ in the direction of a vector field $X:M\to TM$ is the real function defined by
\begin{equation}
  \cL_X f := df(X).
\end{equation}
\end{definition}

From the look of it, this seems just an alternative way to define the directional derivative.
However, its power lies in the fact that we can extend it to $k$-forms with important consequences, one of which will be very useful in the next section.

\begin{definition}[Cartan's Magic Formula]
  Let $M$ be a smooth $n$-manifold and $X\in\fX(M)$.
  For $\omega\in\Omega^k(M)$, we define the \emph{Lie derivative} of $\omega$ with respect to $X$ as the $k$-form
  \begin{equation}\label{eq:cartanmagicf}
    \cL_X\omega := \iota_X (d\omega) + d(\iota_X\omega).
  \end{equation}
\end{definition}

Since the exterior derivative raises the degree of the form and the interior product decreases it, the net effect of the formula above, is indeed, the production of a $k$-form, so $\cL_X\omega \in \Omega^k(M)$.

\begin{exercise}
  Show that on functions the definition from~\eqref{eq:cartanmagicf} coincide with the one that  we gave at the beginning of this section.
\end{exercise}

\begin{exercise}\label{exe:liederandwedge}
Show that the Lie derivative is a derivation in the algebra $\Omega^*(M)$ of differential forms, that is, for $\omega,\nu\in\Omega^*(M)$ one has
\begin{equation}
  \cL_X(\omega\wedge\nu) = (\cL_X\omega)\wedge\nu + \omega\wedge(\cL_X \nu).
\end{equation}
\end{exercise}

It is possible to define the Lie derivative in a different way, in terms of the derivative of the pushforward of $\omega$ along the flow of $X$.
Then the definition that we gave above becomes a theorem, which is where the denotation \emph{Cartan's Magic Formula} comes from.

Of course, we can recover the alternative definition as a theorem.
Even though it is a bit impractical for computational purposes, flows are hard to compute, it gives a nice geometric interpretation of the Lie derivative: it describes the change of the differential form $\omega$ in the direction of the flow generated by the vector field $X$.

\begin{theorem}\label{thm:LieDerivativeFlow}
  Let $M$ be a smooth complete $n$-manifold, $X\in\fX(M)$ and $\varphi_t$ its flow.
  Then, for all $\omega\in \Omega^*(M)$, one has
  \begin{equation}
    \frac{d}{dt}(\varphi_t^* \omega) = \varphi_t^*\cL_X\omega.
  \end{equation}
\end{theorem}
\begin{proof}
  \newthought{Step I}.
  Thanks to the group properties of the flow, it is enough to prove it for $t=0$.
  Indeed,
  \begin{align}
    \frac{d}{dt}(\varphi_t^* \omega)
    &= \frac{d}{ds}(\varphi_{t+s}^* \omega)\Big|_{s=0} \\
    &= \frac{d}{ds}(\varphi_t^*\varphi_s^*\omega)\Big|_{s=0} \\
    &= \varphi^*_t \frac{d}{ds}(\varphi_s^*\omega)\Big|_{s=0}.
  \end{align}

  \newthought{Step II}.
  We start with $f\in\Omega^0(M) = C^\infty(M)$.
  In local coordinates $(x^i)$, we have
  \begin{align}
    \frac{d}{dt}\Big|_{t=0}\varphi_t^* f(x) 
    &= \lim_{t\to0} \frac{f(\varphi_t(x)) - f(t)}{t} \\
    &= \frac{\partial f}{\partial x^i}\Big|_x X^i(x) \\
    &= df(X)(x) = \cL_X f(x).
  \end{align} 

  \newthought{Step III}. Let $\omega = dx^i \in \Omega^1(M)$, then
  \begin{align}
    \frac{d}{dt}(\varphi_t^* dx^i)\Big|_{t=0}
    &= \frac{d}{dt}(d\varphi_t^* x^i)\Big|_{t=0} \\
    &= d\, \frac{d}{dt}(\varphi_t^* x^i)\Big|_{t=0} \\
    &= dX^i.
  \end{align}
  On the other hand,
  \begin{align}
    \cL_X(dx^i)
    &= \iota_X(ddx^i) + d(\iota_X dx^i)\\
    &= d(\iota_X dx^i) \\
    &= dX^i.
  \end{align}

  \newthought{Step IV}. The statement follows from Theorem~\ref{thm:pullbacksdifferentialforms} and Exercise~\ref{exe:liederandwedge} since every $k$-form can be locally written as $\omega = \omega_I dx^I$.
\end{proof}

\begin{remark}
  The Lie derivative can be extended to any tensor bundle $T_s^r(M)$ with the following definition.
  This $T\in\cT_r^s(M)$, for any $p\in M$
  \begin{equation}
    (\cL_X T)_p := \frac{d}{dt}\Big|_{t=0}\left((\varphi_t^X)^* T\right)_p,
  \end{equation}
  where as usual $\varphi_t^X$ denotes the maximal integral curve\footnote{Remember, this is a diffeomorphims from a neighbourhood of $p$ onto a neighbourhood of $\varphi_t^X(p)$.} for $X$ with initial point $p$.

  In general, for $\tau\in\cT_r^s(M)$ and $\sigma\in\cT_{r'}^{s'}$, the Lie derivative satisfies
  \begin{equation}
    \cL_X(\tau\otimes\sigma) = (\cL_X\tau)\otimes\sigma + \tau\otimes\cL_X(\sigma),
  \end{equation}
  and commutes with contractions.

  Incidentally, it also satisfies
  \begin{equation}
    \cL_XY = [X,Y],
  \end{equation}
  and so it can be considered as a generalization of the Lie brackets.

  One nice little perk of the general definition, is that it makes it relatively straightforward to show that
  \begin{equation}\label{eq:cLXwBrackets}
    \cL_X(\omega(Y)) = (\cL_X\omega)(Y) + \omega([X,Y]),
  \end{equation}
  which is often very useful in computations.

  One can think to the Lie derivative as a mean to ``differentiate'' a tensor field (or a differential form) with respect to a vector field.
  Note that it does not allow us differentiate a tensor field (or a differential form) with respect to a single tangent vector: the value of $\cL_X(\tau)$ at a point depends on the values of $X$ in a neighbourhood of the point, not just on the germ at $X$.
  %This stems from the fact that $X\to\cL_X$ is not $C^\infty(M)$-linear: you can see this by using~\eqref{eq:cLXwBrackets} and show that $\cL_{fX}(\omega)(Y) = f\cL_X(\omega)(Y) + Y(f)\omega(X)$, and in general there is no reason for the correction $Y(f)\omega(X)$ to vanish.
\end{remark}

\section{De Rham cohomology and Poincar\'e lemma}

\begin{definition}
We say that a smooth differential form $\omega\in\Omega^k(M)$ is \emph{closed} if $d\omega = 0$, and \emph{exact} if there exists a smooth $(k-1)$-form $\eta$ on $M$ such that $\omega = d\eta$.

The fact that $d\circ d = 0$ implies that every exact form is closed.
\end{definition}

The following example shows that not all closed forms are exact.
However, it turns out that closed forms are always locally exact but not necessarily globally, so the question of whether a given closed form is exact depends on global properties of the manifold.
This is the statement of the so-called Poincar\'e lemma.
We are going to prove it in two slightly different flavours: its classical version and a slight generalization.

\begin{exercise}[\textit{[homework 4]}]
  Let $M=\R^2\setminus\{0\}$, $\{x,y\}$ denote the standard euclidean coordinates in $\R^2$ and $\omega$ be the one-form on $M$ from Example~\ref{ex:li} given by
  \begin{equation}
    \omega = \frac{xdy - ydx}{x^2+y^2}.
  \end{equation}
  \begin{enumerate}
    \item Show that $\omega$ is closed.
    \item Show that $\omega$ is not exact.\\
    \textit{\small Hint: compare Exercise~\ref{exe:FTC}.3 and Example~\ref{ex:li}.}
  \end{enumerate}
\end{exercise}

\begin{definition}
We define \emph{$k$th de Rham cohomology group} the quotient vector space defined by
\begin{equation}
  H_{\mathrm{dR}}^k(M) := \frac{\{\mbox{closed $k$-forms on $M$}\}}{\{\mbox{exact $k$-forms on $M$}\}}.
\end{equation}
We will denote the elements of $H_{\mathrm{dR}}^k(M)$ by $[\omega]$, where $\omega$ is a closed $k$-form. Thus, by definition, $[\omega + d\theta] = [\omega]$.
\end{definition}

We will use only elementary facts about de Rham theory in the course, but they play an important role in algebraic topology and mechanics.
The de Rham groups, for example, turn out to be topological invariants.

The following is a direct consequence of Exercise~\ref{ex:smoothpushforward}.

\begin{corollary}
If $F:M\to N$ is a smooth map, then $F^*$ induces a well-defined map $F^*:H_{\mathrm{dR}}^k(N) \to H_{\mathrm{dR}}^k(M)$ (denoted with the same symbol) via $[\omega]\mapsto[F^*\omega]$.
\end{corollary}

Without further ado, let's look at a first version of Poincar\'e lemma on manifolds.
As for all the local concepts we have seen so far, the proof will reduce the problem to a euclidean statement to which we will apply the Poincar\'e lemma that you have seen in multivariable calculus.

\begin{theorem}
  Let $M$ be a smooth manifold and $\omega\in\Omega^k(M)$ closed, that is, $d\omega = 0$.
  Let $U\subset M$ be open and diffeomorphic to a star-shaped domain\footnote{Cf. Lemma~\ref{lem:Taylor}.} of $\R^n$.
  Then, there exists $\nu\in\Omega^{k-1}(U)$ such that $\omega|_U = d\nu$.
\end{theorem}
\begin{proof}
Let $\varphi: U \to V\subset\R^n$ be a diffeomorphism between $U$ and the star-shaped domain $V\subset\R^n$.
Then $\widetilde\omega := \varphi_*\omega$ is a closed $k$-form on $V$ and, according to the Poincar\'e lemma on $\R^n$, there exists $\widetilde \nu \in \Omega^{k-1}(V)$ such that $\widetilde\omega = d\widetilde\nu$.
\end{proof}

To generalise this result further, we need to have a deeper look into de Rham theory.

\begin{definition}
  Two continuous maps $h_0, h_1:X\to Y$ between topological spaces are said to be \emph{homotopic} if there exists a continuous map $K: [0,1]\times X\to Y$ such that $K(0, \cdot) = h_0$ and $K(1,\cdot) = h_1$.

  Two topological spaces $X$ and $Y$ are \emph{homotopy equivalent} if there exists continuous maps $f:X\to Y$ and $g:Y\to X$ such that $f\circ g$ and $g\circ f$ are homotopic to the respective identity maps.
\end{definition}

A crucial observation for our means is the \emph{homotopy invariance} of the de Rham cohomology, which is a scary sounding property which is formalised by the following statement.

\begin{theorem}
  Let $M$ be a smooth manifold and $[0,1]\times M$ the product manifold with boundary $(\{0\}\times M) \cup (\{1\}\times M) \cup ((0,1)\times\partial M)$.
  Let $i_t : M \hookrightarrow [0,1]\times M$ be the injection $i_t(p) := (t,p)$ and $\pi : [0,1]\times M \to M$ the projection onto $M$.
  Then, there is a map
  \begin{equation}
    K : \Omega^\ell([0,1]\times M)\to \Omega^{\ell-1}(M)
  \end{equation}
  such that for every differential $\ell$-form $\omega\in\Omega^\ell([0,1]\times M)$ one has
  \begin{equation}
    K(d\omega) + d(K(\omega)) = j^*_1(\omega) - j^*_0(\omega)
  \end{equation}
  as elements of $\Omega^\ell(M)$.
  Furthermore, the induced maps on the de Rham cohomology
  \begin{equation}
    j_0^*, j_1^* : H_{\mathrm{dR}}^\ell([0,1]\times M) \to H_{\mathrm{dR}}^\ell(M)
  \end{equation}
  coincide.
\end{theorem}
\begin{proof}
  Let $T$ be the vector field on $[0,1]\times M$ whose value at $(t,p)$ is given by
  \begin{equation}
    T(t,p) = \left(\frac{\partial}{\partial t}\Big|_{t=0}, 0\right).
  \end{equation}
  Then, for $\omega\in\Omega^\ell([0,1]\times M)$, the map $K$ is defined by
  \begin{equation}
    K(\omega) := \int_0^1 j^*_t(\iota_T(\omega)) dt.
  \end{equation}
  That is, for any $p\in M$,
  \begin{equation}
    K(\omega)_p = \int_0^1 j_t^*(\iota_T(\omega)_{(t,p)})dt,
  \end{equation}
  where the integrand should be thought as a function of $t$ on the vector space $\Lambda^{\ell-1}(M)$.
  That is, this is still a common integral, not an integral on a manifold!
  By choosing local coordinates on $M$, we see that the integral is defining a smooth $(\ell-1)$-form on $M$.
  To compute $d(K(\omega))$ pick some local coordinates $(x^i)$, then we can express $K(\omega)$ as a sum of terms of the form
  \begin{equation}
    \left(\int_0^1 f(t,x) dt\right)dx^I.
  \end{equation}
  Applying the exterior derivative and differentiating under the integral sign\footnote{Also known as Leibniz integral rule and Feynman's trick.} we get
  \begin{equation}
    \frac{\partial}{\partial x^j}\left(\int_0^1 f(t,x) dt\right)dx^j\wedge dx^I = \left(\int_0^1 \frac{\partial f}{\partial x^j}(t,x) dt\right)dx^j\wedge dx^I.
  \end{equation}
  That is,
  \begin{equation}
    d(K(\omega)) = \int_0^1 d(j_t^*(\iota_T(\omega)))dt.
  \end{equation}
  Then, it follows from Cartan's Magic Formula and Exercise~\ref{ex:smoothpushforward} that
  \begin{align}
  K(d\omega) + d(K(\omega))
  &= \int_0^1\left( j_t^*(\iota_T(d\omega)) + d (j_t^*(\iota_T\omega)) \right) dt \\
  &= \int_0^1\left( j_t^*(\iota_T(d\omega)) + j_t^* (d(\iota_T\omega)) \right) dt \\ 
  &= \int_0^1 j_t^*(\cL_T(\omega)) dt.
  \end{align}

  Let $\varphi_t$ now denote the flow of $T$, then $\varphi_t(s, p) = (t+s, p)$ and thus $j_t = \varphi_t \circ j_0$.
  By Theorem~\ref{thm:LieDerivativeFlow} we can compute the integrand as
  \begin{align}
    j_t^*(\cL_T(\omega)) &= j^*_0(\varphi_t^*(\cL_T(\omega))) \\
    &= j_0^*\left(\frac{d}{dt} \varphi_t^*(\omega)\right) \\
    &= \frac{d}{dt} j_0^*(\varphi_t^*(\omega)) \\
    &= \frac{d}{dt} j_t^*(\omega).
  \end{align}

  Thus, by the classical Fundamental Theorem of Calculus we get
  \begin{equation}
  K(d\omega) + d(K(\omega)) = \int_0^1 \frac{d}{dt} j_t^*(\omega) dt = j_1^*(\omega) - j_0^*(\omega),
  \end{equation}
  proving the first part of the theorem.

  To conclude the proof, take a closed $\ell$-form on $[0,1]\times M$, then
  \begin{equation}
    j_1^*([w]) - j_0^*([w]) = [K(d\omega) + d(K(\omega))] = 0,
  \end{equation}
  completing the proof.
\end{proof}

An important consequence of this result is the following theorem.

\begin{theorem}
Let $M$ and $N$ two smooth manifolds and suppose $F,G: M\to N$ are two homotopic smooth maps.
Then, the induced maps $F^*$ and $G^*$ on the de Rham cohomology groups are the same.
\end{theorem}
\begin{proof}
Since $F$ and $G$ are homotopic, there is a continuous map $K: [0,1]\times M \to N$ such that $K(0,\cdot) = F$ and $K(1,\cdot) = G$.
If we could assume $K$ to be smooth, the theorem would follow from
\begin{equation}
  F^* = (H\circ j_0)^* = j_0^*\circ H^* = j_1^*\circ H^* = (H\circ j_1)^* = G^*.
\end{equation}

In fact this is the case, thanks to the Whitney Approximation Theorem for homotopies\footnote{This is a deep result related to the Whitney Embedding Theorem from Remark~\ref{rmk:WhitneyET} and is out of the scope of our course, for more details refer to~\cite[Chapter 6 and Theorems 6.26 and 9.27]{book:lee}.} which says that if two smooth maps are homotopic then they are also smoothly homotopic, in the sense that the map $K$ is smooth.
\end{proof}

\begin{corollary}\label{cor:deRhamIso}
  Let $M$ and $N$ be smooth manifolds that are homotopy equivalent. Then $M$ and $N$ have isomorphic de Rham cohomology groups.
\end{corollary}
\begin{proof}
  Let $F:M\to N$ and $G:N\to M$ be continuous maps such that $F\circ G$ and $G\circ F$ are homotopic to the identity maps. 
  By the Whitney Approximation Theorem (see the proof above) we can approximate $F$ and $G$ by smooth maps that we keep denoting with the same symbols.
  By the previous theorem, then, $(F\circ G)^*$ and $(G\circ F)^*$ coincide with the maps induced by the identity.
  Since $\id^*$ is clearly the identity, we see that $F^*$ is an inverse to $G^*$, which concludes the proof.
\end{proof}

We are almost there.

\begin{definition}
  A topological space is said to be \emph{contractible} if it is homotopy equivalent to a point, that is, there exists $p_0\in M$ and a continuous\footnote{In fact, we now know that we can assume it is smooth.} map
  \begin{equation}
    K:[0,1]\times M \to M
    \quad\mbox{with}\quad
    K(0, \cdot) = \id_M
    \mbox{ and }
    K(1, \cdot) = p_0.
  \end{equation}
  \marginnote[-2em]{The map $K$ continuously ``contracts'' $M$ into a single point $p_0\in M$.}  
\end{definition}

\begin{corollary}
  Let $M$ be contractible, then $H^k_{\mathrm{dR}}(M)=0$ for all $k\geq 1$.
\end{corollary}
\begin{proof}
The statement is clear is $M$ is equal to a point.
The rest follows applying Corollary~\ref{cor:deRhamIso}.
\end{proof}

\begin{remark}
De Rham cohomology is defined in terms of spaces of differential forms and, as such, seems a priory deeply tied to the differential structure.
However, the corollary that we just proved is all about topology and in particular tells us that de Rham cohomology cannot see the smooth structure on a topological manifold.
Indeed, the cohomology cannot distinguish Euclidean spaces since $H_{\mathrm{dR}}(\R^n)$ is independent of $n$. 
\end{remark}

Finally, we are ready to show a more general version of the Poincar\'e lemma as promised.

\begin{corollary}[Poincar\'e lemma]\label{cor:plemma}
  Let $M$ be a smooth manifold and let $\omega\in\Omega^k(M)$ be a closed differential form of positive degree $k>0$.
  For any point $p\in M$ there exists a neighbourhood $U$ of $p$ such that $\omega|_U$ is an exact form in $\Omega^k(U)$.
\end{corollary}
\begin{proof}
Every point in a $n$-manifold has a neighbourhood which is homeomorphic to $\R^n$ and so is contractible.
\end{proof}