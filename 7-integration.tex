We finally have all the main ingredients to generalize our line integral detour and discuss integration of $n$-forms over $n$-dimensional manifolds.

\section{Orientation}

\newthought{We know from calculus one}, or our line integral examples, that the direction in which we traverse the interval, or a curve, can actually make a difference.
Indeed, the sign of the integral of a differential $n$-form is only fixed after choosing an orientation of the manifold.

If for a curve an orientation is simply a choice of a direction along it, so we can make sense of it in terms of clockwise or counter-clockwise, generalising the concept will require an extra abstraction step.
Not just that, you have seen already that in $\R^n$ there is a standard orientation, but in other vector spaces we may need to make arbitrary choices.
For manifolds, the situation is much more complicated: for example, on a M\"obius strip it is impossible to make any such choice, as it turns out, it is non-orientable.

\section{Stokes' Theorem}
