\newthought{The dual of a vector space} should be a well-known concept from linear algebra. We recall it here just for the sake of convenience.

\begin{definition}
  Let $V$ a vector space of dimension $n\in N$.
  Its \emph{dual space} $V^* := \cL(V, \R)$ is the $n$-dimensional real vector space of linear maps $\omega:V \to R$.
  The elements of $V^*$ are usually called \emph{linear functionals} and for $\omega\in V^*$ and $v\in V$ it is common to write
  \begin{equation}
    \omega(v) =: (\omega, v) =: (\omega \mid v),
  \end{equation}
  even if the \emph{dual pairing} $(\omega \mid v)$ is \emph{not} a scalar product.
\end{definition}

\begin{remark}\label{rmk:identification}
  Note that a scalar product $\langle,\rangle :  V\times V \to \R$ on a vector space $V$ provides a natural identification of $V$ and $V^*$ via the map $V\ni v \mapsto \langle v, \cdot \rangle =: \omega(\cdot) \in V^*$.
  Even though $\dim V = \dim V^*$ in any case, without the scalar product there is no such canonical isomorphism.
\end{remark}

In the previous chapter we defined the tangent space as a special vector space over each point in a manifold, which nicely fits in the requirements above.

\begin{definition}
  Let $M$ be a differentiable manifold and $p\in M$.
  The dual space $T_p^*M := (T_pM)^*$ of the tangent space $T_pM$ is called the \emph{cotangent space} of $M$ at $p$.
  The elements of $T^*_pM$ are called \emph{cotangent vectors}, \emph{covectors} or \emph{(differential) $1$-forms} at $p$.
\end{definition}

For a function $f:\R^n\to\R$, we usually consider the gradient $\nabla f(x)$ at a point $x$ to be a vector, on a manifold however things a slightly different.

\begin{example}[The differential of a function]
  Let $f\in C^\infty(M)$. Let's look carefully at its differential:
  \begin{equation}
    df_p: T_p M \to T_{f(p)}\R \simeq \R
  \end{equation}
  is a linear function from the tangent space to $\R$.
  In other words, $df_p \in T_p^* M$.
\end{example}

Whereas tangent vectors give us a coordinate-free interpretation of derivatives (of curves), it turns out that derivatives of real-valued functions on a manifold are most naturally interpreted as cotangent vectors.

Indeed, we saw that the action of $df_p$ on a tangent vector $v\in T_p M$ can be interpreted as the directional derivative of $f$ at $p$ in the direction $v$ and, by using Definition~\ref{def:tg:ascurvespeed}, we have
\begin{equation}
  df_p(v) = \frac{d}{dt}f(\gamma(t))\Big|_{t=0}
\end{equation}
for some curve $\gamma$ with $\gamma(0) = p$ and $\gamma'(0)=v$.
We also know that the equation above can be rewritten by thinking of $v$ as a derivation, giving
\begin{equation}
  df_p(v) = v(f).
\end{equation}
That is, we can think of the dual pairing $(df\mid v)$ in a twofold way:
\begin{itemize}
  \item as a linear action of the covector $df$ on the vector $v$;
  \item as the linear action of the vector $v$ as a derivation operating on the function $f$.
\end{itemize}

\begin{notation}
  In analogy to the notation $\frac{\partial}{\partial x}\big|_p$ that we used for tangent vectors, when convenient we may write $df|_p$ instead of $df_p$.
\end{notation}

To look more concretely at differential forms, let's compute its coordinate representation.
Let $(U,\phi)$ be a chart on $M^n$ with local coordinates $(x^i)$.
Since the coordinate functions $x^i\in C^\infty(U)$, we can define the corresponding coordinate $1$-forms $dx^i|_p \in T_p^* M$.
Their action on the coordinate vector fields, then, is immediately computed as
\begin{equation}
  \left(dx^i|_p ,\; \frac{\partial}{\partial x^j}\Big|_p\right) =
  dx^i|_p \left(\frac{\partial}{\partial x^j}\Big|_p\right)
  = \frac{\partial}{\partial x^j}\Big|_p x^i
  = \delta^i_j.
\end{equation}
Which proves the following statement.

\begin{proposition}
  Let $(x^i)$ be local coordinates on an open subset $U\subseteq M^n$.
  At each point $p\in U$, the covectors $(dx^1|_p, \ldots, dx^n_p)$ form a basis for the cotangent space $T_p^* M$ which is dual to the basis $\frac{\partial}{\partial x^1}\Big|_p, \ldots, \frac{\partial}{\partial x^n}\Big|_p$ for the tangent space $T_p M$.
\end{proposition}

That is, any $1$-form $\omega$ can be locally written as a linear combination
\begin{equation}
  \omega = \omega_i\; dx^i
\end{equation}
where $\omega_i:U\to\R$.
In particular, if $f\in C^\infty(M)$, the restriction $df$ to points in $U$ should have the same form.
Evaluating it on a coordinate vector field gives, for all $p\in U$,
\begin{align}
  &\underset{\shortparallel}{df_p} \left(\frac{\partial}{\partial x^j}\Big|_p\right)
  = \omega_i\; dx^i|_p \left(\frac{\partial}{\partial x^j}\Big|_p\right) = 
  \omega_i \delta^i_j = \omega_j \\
  & \frac{\partial f}{\partial x^j}(p).
\end{align}

That is, the local expression for $df$ is
\begin{equation}\label{eq:localformofthedifferential}
  df =  \frac{\partial f}{\partial x^i} dx^i.
\end{equation}

\begin{remark}
  In calculus 1 you have probably been told that you can cancel out differentials when applying solving differential equations.
  This was probably accompanied by a warning that it is just a formal thing, a computational convenience.
  We can finally make sense of that in a general context: in one dimension, \eqref{eq:localformofthedifferential}, reads as
  \begin{equation}
    d f = \frac{df}{dt}dt.
  \end{equation}
\end{remark}

With the local basis, computing with covectors becomes much easier.
Given a covector $\omega = \omega_j\; dx^j$ and a vector $v = v^i \frac{\partial}{\partial x^i}$ expressed in the respective coordinate bases for the local coordinates $(x^i)$, by linearity in both arguments the dual pairing takes the form
\begin{equation}\label{eq:localdualpairing}
  (\omega \mid v) =
  \left(\omega_j\; dx^j \;\Big |\; v^i \frac{\partial}{\partial x^i} \right) =
  \omega_j v^i \left(dx^j \;\Big |\; \frac{\partial}{\partial x^i} \right) =
  \omega_j v^j.
\end{equation}

\begin{remark}[The double dual]
  We said in Remark~\ref{rmk:identification} that unless we have an inner product, there is no canonical identification of a vector space with its dual.
  This is true also for the tangent and cotangent spaces.
  However, the situation is different for the double dual $T^{**}_pM := (T^*_pM)^*$.
  
  For $v\in T_p M$, the map
  \begin{equation}
    i_v: T_p^*M\to \R, \qquad
    \omega \mapsto i_v(\omega) := (\omega\mid v)
  \end{equation}
  is linear and therefore $i_v\in T^{**}_pM$.

  Furthermore the map $i : T_pM \to T_p^{**}M$, $i\mapsto i_v$, is a vector space isomorphism. Indeed, it is injective since $\ker(i) = \{0\}$ and since $\dim T_p M = \dim T^{**}_p M$ also surjective (why? I hope you remember the rank-nullity theorem).

  That is, $T^{**}_pM$ can be canonically identified with $T_p M$.

  So, to add up to our list of interpretations of geometric objects, we now have seen that
  \begin{itemize}
    \item a covector can act as a linear functional on vectors;
    \item a vector can act as a linear functional of vectors.
  \end{itemize}
\end{remark}

This should start giving you an idea of what is behind the following famous quote by Henri Poincar\'e:
\begin{quote}
  Mathematics is the art of giving the same name to different things.
\end{quote}

\section{Change of coordinates}

In Remark~\ref{rmk:chg_coords} we have seen that if we have two different charts with local coordinates $(x^i)$ and $(y^i)$ on a smooth manifold $M$,
\begin{equation}
  \frac{\partial}{\partial x^i}\Big|_p = \frac{\partial y^j}{\partial x^i}(p) \frac{\partial}{\partial y^j}\Big|_p.
\end{equation}
Thus, if $v\in T_pM$ has local form $v = v^i \frac{\partial}{\partial x^i}\big|_p = \widetilde v^j \frac{\partial}{\partial y^j}\big|_p$, we get 
\begin{align}
  & \underset{\shortparallel}{v^i \frac{\partial}{\partial x^i}\Big|_p} = v^i \frac{\partial y^j}{\partial x^i}(p) \frac{\partial}{\partial y^j}\Big|_p \\
  & \widetilde v^j \frac{\partial}{\partial y^j}\Big|_p,
\end{align}
or, reading off the basis elements,
\begin{equation}\label{eq:contravariant}
  \widetilde v^j = \frac{\partial y^j}{\partial x^i}(p) v^i.
\end{equation}

Let now $\omega\in T_p^*M$ with local form $\omega = \omega_i dx^i|_p = \widetilde \omega_j dy^j|_p$.
In analogy to our previous computations we get
\begin{equation}
  \omega_i
  = \omega\left(\frac{\partial}{\partial x^i}\Big|_p\right)
  = \omega\left(\frac{\partial y^j}{\partial x^i}(p) \frac{\partial}{\partial y^j}\Big|_p\right)
  = \frac{\partial y^j}{\partial x^i}(p) \widetilde\omega_j.
\end{equation}
That is,
\begin{equation}\label{eq:covariant}
  \omega_i = \frac{\partial y^j}{\partial x^i}(p) \widetilde\omega_j.
\end{equation}

There is an important difference\footnote{I have borrowed this explanation from \cite[Chapter 11]{book:lee}.} between \eqref{eq:covariant} and \eqref{eq:contravariant}.
For covectors, \eqref{eq:covariant} shows that their components transform in the same way as (``vary with'') the coordinate partial derivatives: the Jacobian of the change of variables $\frac{\partial y^j}{\partial x^i}(p)$ multiplies the objects associated to the ``new'' coordinates $y^j$ to obtain the objects associated to the ``old'' coordinates $x^i$.
For this reason covectors are said to be \emph{covariant vectors}.
Analogously, tangent vectors are said to be \emph{contravariant vectors}, since \eqref{eq:contravariant} shows that their components transform in the opposite way.

This difference in the way vector and covector transform is also reflected in the way they are transformed by smooth maps between manifolds.
As we have seen, the differential of a smooth map yields a linear map between tangent spaces, 

\begin{definition}\label{def:pullback:oneform}
  Let $F:M\to N$ be a smooth map and $p\in M$.
  The \emph{pullback} of $F$ at the point $F(p)$, is the map
  \marginnote{Which is also the reason why, some authors, use the notation $F_*$ to denote the differential of maps between manifolds and other call pushforward the differential.}
  \begin{equation}
    F^* : T^*_{F(p)} N \to T^*_p M, \quad \omega \mapsto F^*\omega,
  \end{equation}
  defined by duality in the following way:
  \begin{equation}
    \left(F^*\omega, v\right) := \left(\omega, dF(v)\right),\quad
    \forall v\in T_pM,\; \forall \omega\in T^*_{F(p)}N.
  \end{equation}
\end{definition}

\marginnote{Equations are getting more and more ticky: this kind of dimensional analysis is extremely useful to check that you are doing the right thing.}
Be careful in the equation above: $F^* \omega \in T^*_p M$ so $v\in T_p M$, but $\omega\in T^*_{F(p)}N$ so $df(v)\in T_{F(p)}N$ since $df: T_pM\to T_{F(p)}N$.

\section{One-forms and the cotangent bundle}

In analogy to Chapter~\ref{sec:tangentbundle} we can glue the cotangent space together into a vector bundle on $M$.

\begin{definition}
  The \emph{cotangent bundle} $T^*M$ of $M$ is the disjoint union of cotangent spaces
  \begin{equation}
    TM := \bigsqcup_{p\in M}\left(\{p\}\times T^*_pM\right)
       = \{(p,\omega) \;\mid\; p\in M,\, \omega\in T^*_pM\}.
  \end{equation}
\end{definition}

The cotangent bundle is a vector bundle of rank $n$ with projection $\pi:T^*M\to M$, $(p,\omega)\mapsto p$.
The cotangent spaces are the fibres of the cotangent bundle. 

Since differential forms are dual objects to vector fields, the action of a form $\omega$ on $X\in\fX(M)$ is well defined and pointwise defines a function
\begin{equation}
  (\omega \mid X) : p \mapsto (\omega_p \mid X_p).
\end{equation}
The differential form $\omega$ is smooth if and only if, for every smooth vector field $X\in\fX(M)$, the function $(\omega \mid X)\in C^\infty(M)$.

\begin{definition}
  A \emph{covector field} or a \emph{(differential) $1$-form} on $M$ is a smooth section of $TM$.
  That is, a $1$-form $\omega\in\Gamma(TM)$ is a smooth map $\omega: p \to \omega_p \in T_p^*M$ that assigns to each point $p\in M$ a cotangent vector at $p$.
\end{definition}

Cotangent Bundle + forms as sections of the cotangent bundle